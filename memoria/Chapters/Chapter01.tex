\parindent=0em
\chapter{Introducción}
\pagenumbering{arabic}
\noindent

En esta sección desarrollaremos la motivación del proyecto, los objetivos que se tienen que alcanzar y el plan de trabajo que se ha seguido.

\section{Motivación}
Actualmente, no existe un juez automático para sentencias SQL que funcione de manera precisa, es decir, los algoritmos que podemos encontrar comparan los datos de salida, no la sentencia en sí. Esto permite que algunas correcciones, a pesar de que den un resultado correcto, sean erróneas. Así es, que a día de hoy en las universidades de toda España se practiquen y evalúen los conocimientos sobre bases de datos a papel, lo que supone un gran retraso en la enseñanza de esta materia respecto a otras.

Cogiendo de ejemplo otros jueces de programación, se puede sacar provecho de estos sistemas automáticos de distintas maneras. Para empezar, se podría llevar un registro de los ejercicios realizados por cada alumno, así como de sus aciertos y errores. Otra sería el uso de una herramienta que proporciona resultados al instante, sin tener que esperar a la corrección del profesor. Además, no solo se reduciría la carga de trabajo del profesor, sino que también los propios alumnos se verían beneficiados, ya que el tiempo que el profesor emplearía en la corrección de los ejercicios, lo podría emplear en la resolución de dudas sobre éstos.

\section{Objetivos}
El principal objetivo del proyecto es crear un algoritmo capaz de determinar si dos sentencias SQL son equivalentes entre sí. En concreto tenemos las siguientes metas: 
\begin{itemize}
    \item Traducir las consultas SQL a expresiones del álgebra relacional en formato JSON mediante la librería mo-sql-parsing
    \item Aplicar las reglas del álgebra relacional sobre los JSON traducidos previamente.
    \item Unificar las dos partes anteriores mediante un algoritmo de selección de reglas a aplicar, en función de la consulta introducida.
    \item (Tests)
    \item (Renombramientos)
    \item Realizar todos los aspectos anteriores en un tiempo de respuesta óptimo, de esta manera, si la aplicación tiene muchos datos de entrada a verificar, no saldrán tiempos de espera muy largos.

\end{itemize}

\section{Plan de trabajo}
Se elaboró un plan para el desarrollo del proyecto de modo que se incluían las fases de análisis preliminar del proyecto, investigación de recursos, análisis de requisitos, implementación, \textit{testing} y escritura de la memoria.