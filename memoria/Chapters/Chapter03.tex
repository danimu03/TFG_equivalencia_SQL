\chapter{Preliminares}
\noindent

En este capítulo vamos a profundizar en algunos términos y tecnologías necesarios para entender el funcionamiento de esta biblioteca.

\section{SQL}
SQL (Structured Query Language) es un lenguaje enfocado al uso de bases de datos relacionales para obtener, almacenar o modificar datos.

\section{Consultas conjuntivas}
Son consultas de la forma SELECT FROM WHERE donde las condiciones WHERE son igualdades combinadas con AND.

(Poner una consulta conjuntiva y una no conjuntiva para mayor entendimiento)

\section{Consultas equivalentes}
Dos consultas son equivalentes cuando el resultado de ambas consultas es el mismo, es decir, las columnas resultantes y los datos mostrados son los mismos.
Otra manera de comprobar si dos consultas son equivalentes es ver si las formas canónicas en álgebra relacional de las consultas SQL son iguales.

\section{Renombramiento}
Para simplificar la utilización de varias tablas en una misma consulta se utiliza el renombrado de las mismas. Este método no modifica el nombre de las tablas en la base de datos. Gracias a este método se puede entender más fácilmente la totalidad de la consulta SQL.

\section{Álgebra relacional}

\section{Reglas de simplificación de álgebra relacional}

\section{Learn SQL}

\section{Python}

\section{mo-sql-parsing}